\documentclass{article}

\author{Mukaumba.J.Comfort}
\title{Computer Graphics}
\date{\today}

\begin{document}

\maketitle

\section{Section A (Q1) [10 points]}
\subsection{a}Which of the following is NOT a primary color in the RGB color model(Red,Green,Blue and Yellow)?\par
\underline{Yellow}

\subsection{b}When light travels from air into water, it generally does what?\par
\underline{Slows down and bends toward the normal.}

\subsection{c}In ray tracing,what is the term used to describe the process of determing which objects in the scene are visible from a specific point of viem?\par
\underline{Ray intersection}

\subsection{d}Which rendering technique is primarily used to simulate the way light interacts with translucent materials such as glass or water?\\
\underline{Reflection.}

\subsection{e}What does the term "keyframe" refer to in animation?\\
\underline{A frame that defines the start or end of a motion sequences.}

\subsection{Q2(a)RAY TRACING AND ANIMATION [12 points]}Briefly explain the concept of specular reflection and provide an example of real-world situation where it occurs?\\
Specular reflection is a type of reflection that occurs when light rays strikes a smooth surface and bounce off in a predictable and organised manner.\\
It is characterized by a mirror-like reflection,where the incident angle of the light rays equals the reflected angle.\\
\textbf{Example of Specular Reflection:}\\
Reflection of light on a calm body of water,such as a lake or a pond.When Sunlight or any other light source shines on the water surface, the smooth water acts as a mirror,reflecting the light rays in a clear reflection of the surronding environment on the water's surface,where the angle of incidence equals the angle of reflection.

\subsection{b}Define "ray tracing" and "ray casting" in the context of Computer Graphics?\\ Explain how they differ?\\
\underline{Ray Casting:}is a relatively simpler technique.It involves casting a primary ray through each pixel on the screen and checking for intersections with objects in the scene.\\
By determing the closest intersectionpoint, the color or properties of the object at that can be calculated and assigned to the corresponding pixel on the screen.\\
\underline{Ray Tracing:}is a more advanced and computationally intensive technique that aims to simulate the behavior of light in a scene with greater realism.\\
\textbf{Differences Between Ray Casting And Ray Tracing:}\par
\underline{Ray Casting:}Can simulate complete lighting effects,including realistic reflections,refractions,shadows and global illumination.\par
\underline{Ray Tracing:}On the other hand,is more sophisticated approach that simulates the path of light rays,allowing more realistic renderingwith advanced lighting effects.  

\subsection{c}Describe the term "Interpolation" as it relates to animation and its impotarnce in creating smooth motion?\par
\underline{Interpolation:}is the process of generating intermediate frames between two keyframes to create smooth motion.\\
It involves calculating the values of attributes(such as position,rotation,scale or color)at a specific point in time.\\
\underline{The Impotarnce Of Interpolation Is:}\\
\begin{itemize}
    \item To achieve fuild and seamless motion between key poses or keyframes.
    \item It automatically calculates the in-between frames, where interpolation helps to create the illusion of continuous movement and natural transitions.
\end{itemize}

\subsection{Q3 PROBLEM-SOLVING RAY OF LIGHT(SNELL'S LAW) [10 points]}Consider a ray tracing scenario where a ray of light travels from air(n = 1.00) into a glass block(1.50).The incident angle is 30 degrees,calculate the angle of refraction using snell's law?(Show your working).\\
Fisrt,we need to convert the incident angle from degrees to radians.\\The formula to use when converting degrees to radians is given as:\\
radians = degrees*(thetha/180 degrees).\par
converting 30 degrees to radians: 01(incident angle) = 30*(thetha/180) which gives:\underline{0.5236 radians.}\par
Now,we can use the snell's law to calculate the angle of refraction(using the formula n1(sin(1)) = n2(sin(2))):\par
Plugging in the values:1.00 * sin(0.5236) = 1.50 * sin(2)\\
So, we can solve for 2 by rearranging the equation:\par
sin(2) = (n1/n2)*sin(1)\par
sin(2) = (1.00/1.50)*sin(0.5236)\\
So,sin(2) =\underline{0.6667}\\
Now,we can find the value of 2 by taking the inverse sine(arcsine) of the value we obtained:\\
So value of 2 = arcsine(0.6667) giving:\underline{41.81 degrees.}\\
Therefore, the angle of refraction when a ray of light travels from air(refractive index 1.00) into a glass block(refractive index 1.50) with an incident angle of 30 degrees is approximately to:\underline{41.81 degrees.} 
\end{document}